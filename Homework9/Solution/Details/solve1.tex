\begin{solution}
    假设图$G$的最大流为$f$,构造图$G'$,使得对于$(u,v)\in E$,$(u,v)$在图$G'$中的容量
    $c(u,v)=f(u,v)$。则对$G'$按照不考虑反向边$(v,u)$的\proc{Ford-Fulkerson}算法
    得到的增广路的数目至多$|E|$。
    
    第一步证明$G'$的最大流不受影响:显然,$f$在$G'$中仍然存在,即$f$是$G'$可行的一个流。
    % $G'$对于流$f$的残余网络,由于不考虑反向边的产生,而原有边$(u,v)$可能会消失,
    % 若$G'$对于流$f$的不考虑反向边的残余网络边集为$E'_f$,则相对原图$G'$对于流$f$的剩余网络边集为
    % $E_f$,$E'_f\subseteq E_f$。
    % 由于$f$是$G$的最大流,故$E_f$不存在新的增广路径,于是$E'_f$也不存在新的增广路径。
    另外$G'$对于流$f$的残余网络,显然不存在源点$s$的出边,故$E'_f$不存在增广路径。
    即$G'$不存在比$f$更大的流。\\
    综上,仍然可以得到$f$是$G'$的最大流。

    第二步证明$G'$按照不考虑反向边产生的\proc{Ford-Fulkerson}方法得到的增广路的数目至多$|E|$:首先每次的增广
    路径求解,必然会使得$G'$中至少一条边$(u,v)$,在残余网络之中消失。由于不再考虑反向边,所以
    $(u,v)$一旦消失不再出现,故可以得到至多$|E|$增广路径的序列。                                                                                                                                                                                                                                                                                                                                                                                                                                                                                                                                                                                                                                                                                                                                                                                                                                                                                                                                                                                                                                                                                                                                                                                                                                                                                                                                                                                                                                                                                                                                                                                                                                                                                                                                                                                                                                                                                                                                               
\end{solution}