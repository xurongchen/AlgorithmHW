\begin{proof}\ \\
    \newcommand{\Doo}{\>\textbf{}\hspace*{-0.7em}\'\addtocounter{indent}{1}}
    \resizebox{\linewidth}{!}{ 
        \begin{minipage}{380pt}
            \begin{codebox}
                \Procname{$\proc{Height-First-Discharge}(G,s,t)$}
                \li $\proc{Initialize-Preflow}(G,s)$
                \li \For each vertex $u\in G.V - \{s,t\}$\Doo
                \li     $u.current = u.N.head$
                    \End
                \li $L = $ ordered linked list of vertex in $G$ by height decreasing
                \li $u = L.Head$
                \li \While $u\neq$ NIL \Doo
                \li     $old\_height = u.Height$
                \li     $\proc{Discharge}(u)$
                \li     \If $old\_height < u.height$ \Doo
                \li         Forward swap $u$ with the former node until the order is correct
                \li     \Else 
                \li         $u = u.Next$
            \end{codebox}
        \end{minipage}
    }
    \textbf{正确性证明:}由于是\proc{Push-Relabel}算法的一种实现,因此只需要证明如果算法能够终止,不能够继续\proc{Push}
    和\proc{Relabel},则算法一定可以得到最大流。下面证明在算法终止时不再存在溢出结点,\proc{Push}和\proc{Relabel}即
    不能再适用。

    首先,高度变化只可能在\proc{Relabel}中发生,因而行10保证了$L$的有序性。

    在行6位置,对于当前访问的$u$,$L$中$u$之前位置不存在溢出结点。这是因为:
    (1)初始状态显然满足;
    (2)每次循环时,对于非溢出结点,执行\proc{Discharge}之后仍然是非溢出的。
    对于溢出结点$u$,显然对于$u$之前的某一结点$v$由于$h(v)>=h(u)$,所以不存在$(u,v)$的\proc{Push}操作;而\proc{Relabel}
    操作不改变结点是否溢出的性质。

    于是算法终止时不再存在溢出结点,\proc{Push}和\proc{Relabel}不再适用,算法可以得到最大流。


    \textbf{复杂度证明:}



    % 已知\proc{Relabel-To-Front}算法的时间复杂度为$O(V^3)$,下面证明按照\proc{Relabel-To-Front}算法执行
    % \proc{Discharge}的结点亦是当前最高的溢出结点。

    % 初始状态:所有的溢出结点的高度均为0,故\proc{Relabel-To-Front}执行的结点是最高的

\end{proof}