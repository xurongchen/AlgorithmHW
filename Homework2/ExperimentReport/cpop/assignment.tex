% Homework template for Algorithm Analysis and Design
% UPDATE: September 20, 2019 by Xu Rongchen
\documentclass[a4paper]{article}
\usepackage{ctex}
\ctexset{
proofname = \heiti{证明} %% set proof name
}
\usepackage{amsmath, amssymb, amsthm}
% amsmath: equation*, amssymb: mathbb, amsthm: proof
\usepackage{moreenum}
\usepackage{mathtools}
\usepackage{url}
\usepackage{bm}
\usepackage{enumitem}
\usepackage{graphicx}
\usepackage{subcaption}
\usepackage{booktabs} % toprule
\usepackage[mathcal]{eucal}

\usepackage{iidef} % set homework count
\usepackage{longtable}

% \usepackage[noend]{algpseudocode}
\usepackage{clrscode3e}

\thecoursename{算法分析与设计实验报告}
\theterm{2019年秋季学期}
\hwname{最近点对}
\slname{\heiti{解}}
\begin{document}
\courseheader
\theusername{徐荣琛}
\thestuno{2019214518}
\theinstitute{软件学院}

\info

\begin{enumerate}
  \setlength{\itemsep}{3\parskip}
  %% Homework Start here:
  %% \item to enumerate the problem ID: Format as 'HomeworkID.ProblemID'
  %% \begin{solution} XXXX \end{solution} is to make a solution
  %% \begin{proof} XXXX \end{proof} is to make a proof
  %% Suggest to use \input{path} command
  \textbf{1.实验内容}\\
  
  \textbf{2.实验环境}\\
  实验采用C\#语言实现,具体的实验环境如下表所示:\\~\\
  \begin{tabular}{c|c}
    \hline\hline
    处理器 & Intel(R) Core(TM) i7-8850H CPU @ 2.60GHz \\ \hline
    内存 & 16GB\\ \hline
    操作系统& Mac OS X 10.14.6\\ \hline
    编译环境& .Net Core 3.0.100\\
    \hline\hline
  \end{tabular}\\
  \bigskip
  
  \bigskip
  \textbf{3.用户界面}
\end{enumerate}


\end{document}