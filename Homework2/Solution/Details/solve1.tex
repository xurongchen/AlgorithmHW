\begin{solution}
    (1)每个$A[i]$在B中各位置的概率均为$1/n$:\\
    若$A[i]$最终位置为$B[j]$,则存在一条位置变化路径:
    $i,p_1,p_2,\cdots,p_{n-2},p_{n-1},j$,因为存在$n$次交换且每次交换后在各位置概率均作为$1/n$,
    这条路径的出现概率$(\frac{1}{n})^n$。而其中$p_1,p_2,\cdots,p_{n-1}$均可以任取$1$至$n$,所以一共
    有$n^{n-1}$种这样的路径,故每个$A[i]$在B中各位置的概率均为$\frac{1}{n}$;\\
    (2)非等概率随机的排列:\\
    $n$个位置,每个位置有$n$个交换可能,故一共有$n^n$个等概率结果,而如果要求对排列进行等概率随机,
    则必然要求将$n^n$个等概率结果分配在$n!$个等概率事件中。显然$n^n$不一定能够被$n!$整除,故该算法
    不能得到等概率随机的排列.

    % 以$n=2$为例证明非等概率\\
    % 假设$P(12)$表示$B=A[1]A[2]$的概率,$P(21)$表示$B=A[2]A[1]$的概率.\\
    % 最初,$P(12)=1$,$P(21)=0$;
    


    % 1 0
    % 1/2 1/2
    % 1/4+3/8 3/8

\end{solution}