Binary search of a sorted array takes logarithmic search time, but the 
time to insert a new element is linear in the size of the array. We can 
improve the time for insertion by keeping several sorted arrays.\\
Specifically, suppose that we wish to support \proc{Search} and 
\proc{Insert} on a set of $n$ elements. Let $k = \lceil \lg (n+1) \rceil$, 
and let the binary representation of $n$ be $\langle n_{k-1}, n_{k-2}, ...,n_0 \rangle$. 
We have $k$ sorted arrays $A_0,A_1,...,A_{k-1}$, where for 
$i=0,1,...,k-1$, the length of array $A_i$ is $2^i$. Each array is either
full or empty, depending on whether $n_i = 1$ or $n_i = 0$, respectively. 
The total number of elements held in all $k$ arrays is therefore 
$\sum_{i=0}^{k-1}{n_i2^i}=n$. Although each individual array is sorted, 
elements in different arrays bear no particular relationship to each other.

\textbf{a.} Describe how to perform the \proc{Search} operation for this data 
structure. Analyze its worst-case running time.

\textbf{b.} Describe how to perform the \proc{Insert} operation. Analyze its worst-case and amortized running times.

\textbf{c.} Discuss how to implement \proc{Delete}.