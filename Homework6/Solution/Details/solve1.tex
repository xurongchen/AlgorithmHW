\begin{solution}
\textbf{a.}
假设所有任务集合为$S$,提前任务的最优独立集合为$A$,则延后任务集合为$S-A$。由于任务调度问题可以
转化为拟阵问题,所以要证明算法的正确性,只需要证明算法能够达到以下的策略:按照惩罚排序后的任务序列,
对于当前任务$a_j$,若$A\cup \{a_j\}$仍然独立,则$A = A\cup \{a_j\}$。

如果对于任务$a_j$的要求结束时间$d_j$前不存在空余时间,$A\cup \{a_j\}$不独立,此时的算法操作
符合$a_j \in S-A$;\\
如果对于任务$a_j$的要求结束时间$d_j$前存在空余时间,$A\cup \{a_j\}$独立,此时的算法操作
符合$a_j \in A$。若空余空间唯一,则显然只有一种策略,即为最优策略;若不唯一,假设一个最优策略是
$a_j$放在了$b$位置,而存在$c$,满足$b < c \le d_j$仍是空余空间。对于后续某个任务$a_w$,若
其要求结束时间$d_w \ge c$,显然亦有$d_w \ge b$,即如果$a_j$放在$b$位置是最优策略,那么放在$b$位
置更靠后的位置亦是最优策略。所以算法选择满足条件的最后位置一定是最优的。

于是上述算法可以得到最优解。

\textbf{b.}
算法过程如\proc{Schedule}所示,默认的$S$的任务的结束时间均能够在$S.size$内,否则必然无法完成,
可直接放入延后任务集合。首先任务集合$S$按照惩罚值降序排序,并初始化结果数组。之后初始化非相交集合
森林的数据结构(路径压缩的实现),每个结点指向的根结点表示前一个(或自身)空余时间位置。初始时,各结点指向自身。
特别的,额外设置0号结点,若某个结点指向的根结点为0号结点,则说明这个结点之前(包括自身)不存在空余位置。
注意,算法中的每次合并$\proc{Union}(x,y)$总是将$x$的根结点指向$y$的根结点,以保证每个等价类的代表元即
第一个空余的时间。
\newcommand{\Doo}{\>\textbf{}\hspace*{-0.7em}\'\addtocounter{indent}{1}}
\begin{codebox}
    \Procname{\proc{Schedule}($S$)}
    \li Sort $S$ by $S[i].penalty$ decreasely
    \li Initialize array $A$ with size of $S.size$
    \li $\proc{Make-Set}(0)$
    \li \For $x = 1$ \To $S.size$ \Doo
    \li     $\proc{Make-Set}(x)$ \End
    \li \For $i = 1$ \To $S.size$ \Doo
    \li     $f = \proc{Find-Set}(S[i].d)$
    \li     \If $f == 0$ \Doo
    \li         $last = \proc{Find-Set}(S.size)$
    \li         $A[last] = S[i]$
    \li         $\proc{Union}(last,last - 1)$
    \li     \Else 
    \li         $A[f] = S[i]$
    \li         $\proc{Union}(f,f - 1)$ \End \End \End
    \li \Return $A$
\end{codebox}

算法中,排序的复杂度为$\Theta(n\log n)$,每次循环之中,\proc{Make-Set}、\proc{Find-Set}
以及\proc{Union}的时间都是$O(1)$的。所以总体算法的复杂度是$\Theta(n\log n)$。
\end{solution}