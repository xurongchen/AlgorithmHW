Consider the following algorithm for the problem from Section 16.5 
of scheduling unit-time tasks with deadlines and penalties. Let all 
$n$ time slots be initially empty, where time slot $i$ is the unit-length 
slot of time that finishes at time $i$. We consider the tasks in order 
of monotonically decreasing penalty. When considering task $a_j$, 
if there exists a time slot at or before $a_j$'s deadline $dj$ that is still
empty, assign $a_j$ to the latest such slot, filling it. If there is no 
such slot, assign task $a_j$ to the latest of the as yet unfilled slots.

\textbf{a.} Argue that this algorithm always gives an optimal answer.

\textbf{b.} Use the fast disjoint-set forest presented in Section 21.3 to 
implement the algorithm efficiently. Assume that the set of input 
tasks has already been sorted into monotonically decreasing order by 
penalty. Analyze the running time of your implementation.