\begin{solution}由于二分查找存在多种具体的任务类型,例如查找给定值是否存在、查找第一个大于等于
    给定值的数等,为了方便讨论,以下分析均按照查找第一个大于等于给定值的数(不存在则返回空)为例
    进行讨论。

\textbf{a.}
\proc{Search}操作的主要过程即对$k$个子数组序列进行二分查找操作,分别查出第一个大于等于给定值
的至多$k$个数,然后对于这$k$个数再线性求出最小值,所得结果即全体数中第一个大于等于给定值的数。

在一般的情况下(二分查找耗时为$\log$级的时间消耗),时间复杂度为:
\resizebox{\linewidth}{!}{
    \begin{minipage}{330pt}
    $$\sum_{i=1}^{\lceil\log (n+1)\rceil} O(\log {2^{i-1}}) + O(\lceil\log (n+1)\rceil) = O(\log^2 n) + O(\log n) = O(\log^2 n)$$
    \end{minipage}
}

而在最坏情况下(即给定数是最大或者最小的),此时二分查找耗时线性,算法的复杂度为:
$$\sum_{i=1}^{\lceil\log (n+1)\rceil} O(2^{i-1}) + O(\lceil\log (n+1)\rceil) = O(n) + O(\log n) = O(n)$$

\textbf{b.}
\proc{Insert}操作向$A_0$中插入目标数,并按需调整使得$A_0$有序。随后从$i=0$递增开始循环以下过程:\\
(1) 若$A_i$的长度超过$2^i$的上限,则将$A_i$和$A_{i+1}$进行归并排序,结果保存到$A_{i+1}$并清空$A_{i}$;\\
(2) 若$A_i$的长度小于等于$2^i$,则结束循环。

\textit{最坏情况的时间分析:}
当$n=2^m-1, m \in Z^*$时,由于所有的子数组序列均已满,再增加新元素时,需要进行最多次数的
归并排序(若将向$A_0$的插入也计入,共$m$次)。此时的时间复杂度为:
$$\sum_{m=0}^{\log (n+1)} O(2^{m}) = O(n)$$

\textit{均摊分析:}
第$i$次操作前元素个数为$n'$,假设$t$满足$n'_t = 0 \wedge \forall t'<t, n'_{t'} = 1$。此时实际归并总花费
$c_t$有:
$$c_t = \sum_{j=0}^t 2^j = O(2^{t})$$
对于$c_t$,实际出现的频率为$\frac{n}{2^i}$。因此,从0个元素增至$n$个元素,总的花费为:
$$\sum_{i=1}^k O(2^{t}) \cdot \frac{n}{2^i} = O(kn) = O(n\log n)$$
故均摊到每次的花费为:$\frac{O(n\log n)}{n} = O(\log n)$

% 对于$n$的二进制表示$\langle n_{k-1}, n_{k-2}, ...,n_{0}\rangle$约定势函数:
% $$\Phi(n) = \sum_{p \in \{p \mid n_p = 1\}} 2^p$$

% 第$i$次操作前元素个数为$n'$,假设$t$满足$n'_t = 0 \wedge \forall t'<t, n'_{t'} = 1$。此时实际归并总花费
% $c_i$有:
% $$c_i = \sum_{j=0}^t 2^j = O(2^{t})$$
% 势函数差:
% $$\Phi_i - \Phi_{i-1} = $$
% \begin{align*}
%     \hat{c_i} &= c_i + \Phi_i - \Phi_{i-1}\\
%     &= 
% \end{align*}
\textbf{c.}
设\proc{Delete}前元素个数为$n'$,$t$满足$n'_t = 1 \wedge \forall t'<t, n'_{t'} = 0$。
\proc{Delete}首先使用\proc{Search}找到元素$a$的位置(默认元素存在):\\
(1)如果$a \in A_t$,则将$A_t$除去
$a$的其它元素按顺序放在$A_{t-1},A_{t-2},...,A_0$,总的算法复杂度为$O(2^t)$;\\
(2)如果$a \in A_s \wedge s > t$,则将$A_t$中的首个元素取出,放入到$A_s$中,交换顺序直至得到正确位置;再对$A_t$
执行类似(1)的操作。总的复杂度为$O(2^s+2^t) = O(2^s)$。

综上,按上述方法实现的\proc{Delete}复杂度为$O(2^k) = O(n)$。
\end{solution}


% n次,1=》 n/2
% 2=》 n/4
% 3=》 n/8
% 4=》 n/16

% $\sum$

% 第一个0的位置为p,$c_i=\sum_{m=1}^{p} O(2^{m}) =2^p$

% 势函数:1的位置c, $\sum 2^c$