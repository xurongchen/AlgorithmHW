As stated, in dynamic programming we first solve the subproblems and then 
choose which of them  to use in an optimal solution to the problem. 
Professor Capulet claims that we do not always need to solve all the subproblems 
in order to find an optimal solution. She suggests that we can find an optimal 
solution to the matrix-chain multiplication problem by always choosing the 
matrix $A_k$ at which to split the subproduct $A_iA_{i+1}\cdots A_j$
(by selecting $k$ to minimize the quantity $p_{i-1}p_kp_j$) before solving the 
subproblems. Find an instance of the matrix-chain multiplication problem 
for which this greedy approach yields a suboptimal solution.