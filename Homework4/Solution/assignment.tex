% Homework template for Algorithm Analysis and Design
% UPDATE: September 20, 2019 by Xu Rongchen
\documentclass[a4paper]{article}
\usepackage{ctex}
\ctexset{
proofname = \heiti{证明} %% set proof name
}
\usepackage{amsmath, amssymb, amsthm}
% amsmath: equation*, amssymb: mathbb, amsthm: proof
\usepackage{moreenum}
\usepackage{mathtools}
\usepackage{url}
\usepackage{bm}
\usepackage{enumitem}
\usepackage{graphicx}
\usepackage{subcaption}
\usepackage{booktabs} % toprule
\usepackage[mathcal]{eucal}

\usepackage[thehwcnt = 4]{iidef} % set homework count
\usepackage{longtable}

% \usepackage[noend]{algpseudocode}
\usepackage{clrscode3e}

\thecoursename{算法分析与设计}
\theterm{2019年秋季学期}
\hwname{作业}
\slname{\heiti{解}}
\begin{document}
\courseheader
\theusername{徐荣琛}
\thestuno{2019214518}
\theinstitute{软件学院}

\info

\begin{enumerate}
  \setlength{\itemsep}{3\parskip}
  %% Homework Start here:
  %% \item to enumerate the problem ID: Format as 'HomeworkID.ProblemID'
  %% \begin{solution} XXXX \end{solution} is to make a solution
  %% \begin{proof} XXXX \end{proof} is to make a proof
  %% Suggest to use \input{path} command
  \item Prove: $2n+\Theta(n^2) = \Theta(n^2)$.
  \begin{solution}
    首先$O(m)$地计算$P$的前缀函数$\pi$。
    然后初始化$\delta$为0,再外层从$0$到$m$循环$q$,内层循环$\Sigma$中每个字母$a$,若
    \begin{enumerate}
        \item $q < m \wedge P[q+1] = a$,则$\delta(q,a) = q+1$;
        \item $q = m \vee P[q+1] \neq a$,则$\delta(q,a) = \delta(\pi(q),a)$;
    \end{enumerate}
    \begin{proof}

            % (a) 根据自动机状态的定义,该式显然成立;
            % (b) 
            
            根据自动机状态的定义,自动机状态编号即为相同的$T$的后缀在$P$前缀上匹配的最大长度,
            因此假设集合$F(q,a)=\{k\mid P_k \sqsupset P_q \wedge P[k+1] = a\}$ 
            我们希望证明:
            $$\delta(q,a) = \max_k F(q,a) \text{ if } F \neq \emptyset \text{ else } 0$$
            
            对$q$进行归纳:
            
            (1)初始,$\delta(0,x)=0 \text{ if } x \neq P[1] \text{ else } 1$,由
            (a)和初始化条件,显然成立。
            
            (2) 假设对$q<q_0$成立,考虑$F(q,a)$的等价表示:
            $F(q,a)=\{k\mid P_k \sqsupset P_q \wedge P[k+1] = a\} = \{k\mid k \in \pi^*(q) \wedge P[k+1] = a\}$

            若$q < m \wedge P[q+1] = a$,显然$\max F(q,a) = q$,于是$\delta(q,a) = q+1$;
            若$q = m \vee P[q+1] \neq a$,显然$q \notin F(q,a)$,于是$F(q,a) = \max\{k\mid k \in \pi^*(q) \wedge P[k+1] = a\}=\max\{k\mid k \in \pi^*(\pi(q)) \wedge P[k+1] = a\}=\delta(\pi(q),a)$

           综上(a)、(b)得证。
            % 根据$q$的长度进行归纳:
            % % $\delta(\pi(q),a) = \pi(q)+1$
            
            % 初始$q=0$,显然成立;\\
            % % 假设$q<q_0$成立:
            % 由于$P_{\pi(q)}\sqsupset P_q$,于是$P_{\pi(q)}a\sqsupset P_qa$,又
            % $\delta(q,a) = \sigma(P_qa) = \max\{k\mid P_k \sqsupset P_qa\}$。
            % 因此,$\delta(\pi(q),a)\in \{k\mid P_k \sqsupset P_qa\}$。

            % 下面证明是最大的。

            % 由于以$a$结尾,不妨设存在另一个最大的$\delta(l,a)>\delta(\pi(q),a)$,且$P_la \sqsupset P_qa$。
            % 显然,$l$满足$l\in\{l\mid l<q \wedge P_l\sqsupset P_q\}$,
            % 由于$l>\delta(l,a)-1>\delta(\pi(q),a)$

            % 于是$\pi(q) = \max\{t\mid t<q \wedge P_t\sqsupset P_q\} = \max\{t\mid t<q \wedge P_ta\sqsupset P_qa\} = $
            
            % 由于$q = m \vee P[q+1] \neq a$,故$\sigma(P_qa) < P_q + 1 = q + 1$。
            % 而$\pi(q) = \max\{t\mid t<q \wedge P_t\sqsupset P_q\} = \max\{t\mid t<q \wedge P_ta\sqsupset P_qa\} = $,
            % 假设最值为$t_m$,
            % 则$P_{t_m}\sqsupset P_q$
    \end{proof}
\end{solution}
  \item As stated, in dynamic programming we first solve the subproblems and then 
choose which of them  to use in an optimal solution to the problem. 
Professor Capulet claims that we do not always need to solve all the subproblems 
in order to find an optimal solution. She suggests that we can find an optimal 
solution to the matrix-chain multiplication problem by always choosing the 
matrix $A_k$ at which to split the subproduct $A_iA_{i+1}\cdots A_j$
(by selecting $k$ to minimize the quantity $p_{i-1}p_kp_j$) before solving the 
subproblems. Find an instance of the matrix-chain multiplication problem 
for which this greedy approach yields a suboptimal solution.
  \begin{solution}
    若要求点在桶中实现均匀分布,则需要将单位圆按面积等分为$n$个同心圆环(最内部为圆)。\\
    $$\pi r_0^2 \le \frac{\pi}{n}$$
    $$\frac{i\pi}{n} < \pi r_i^2 \le \frac{(i+1)\pi}{n}, i\in [1,n-1]$$
    即$ nr_0^2 \le 1$,$ i < nr_i^2 \le i+1$\\
    按$n(x^2+y^2)$为关键字进行桶排序,$n$个桶为$[0,1],(1,2],(2,3],\cdots,$ $(n-1,n]$.
    这样可得到$\Theta(n)$复杂度的排序算法.
\end{solution}

\end{enumerate}


\end{document}