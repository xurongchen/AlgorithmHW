\begin{solution}
    \textit{算法描述:}首先对活动按照开始时间从大到小排序,初始化选择的活动集合$A$
    为空,然后对排完序的序列逐个进行访问,如果$A$中最早的活动开始
    时间大于等于当前访问的活动$a_i$的结束时间,则将$a_i$加入到$A$中,
    并更新$A$中最早的活动开始时间。访问所有的序列完毕时,得到的集合$A$
    即为最优解;

    \textit{证明:}

    贪心选择性的证明:

    欲证:对于任意一个非空子问题$A_k$,若$a_m$是$A_k$中活动开始时间最晚的,则
    $a_m$在$A_k$的某个最大兼容活动子集之中。

    假设$B_k$是$A_k$的一个最大兼容活动子集,且$a_t$是其中活动开始时间最晚的。
    若$a_t=a_m$,显然原问题已经直接得证;若$a_t \neq a_m$,则集合
    $B'_k = B_k - \{a_t\} \cup \{a_m\}$,因为$B_k$的活动是不相交的,且
    $a_t$的开始时间$s_t$和$a_m$的开始时间$s_m$有:$s_m \ge s_t$,显然
    $B'_k$的活动也是不相交,而$|B'_k| = |B_k|$,故$B'_k$也是$A_k$的一
    个最大兼容活动子集,得证。

    最优子结构的证明:

    假设$S_{ij}$是活动$a_i$结束之后开始,$a_j$开始之前结束所有活动的集合,$A_{ij}$是
    $S_{ij}$的一个包括$a_k$的最大兼容活动子集,则对于$A_{ij}$有以下式子成立:
    $$A_{ij} = A_{ik} \cup \{a_k\} \cup A_{kj}$$
    $$|A_{ij}| = |A_{ik}| + |A_{kj}| + 1$$

    显然,$A_{ik}$是$S_{ik}$最大兼容活动子集,否则存在$A'_{ik}$,有$|A'_{ik}|<|A_{ik}|$,
    显然存在$A'_{ij} = A'_{ik} \cup \{a_k\} \cup A_{kj}$,使得$|A'_{ij}|<|A_{ij}|$,
    如此产生了矛盾。同理$A_{kj}$是$S_{kj}$最大兼容活动子集。最优子结构性质得证。

    结合贪心选择性和最优子结构性质,算法得证。
\end{solution}