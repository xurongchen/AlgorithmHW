\begin{solution}\textit{算法描述:}算法$\proc{Huffman-Ternary}$如下伪代码所示,在此之中$\proc{Extract-Min}$与二进制的
    哈夫曼编码算法原有实现一致,即从集合中遍历找到出现频率最小的,删除这个元素并返回;$\proc{Insert}$
    亦与原有实现一致,即向集合中插入一个元素。

    算法的大致思路是:最初如果总元素个数为偶数个,则从中选两个结点合并(减少一个结点)。之后元素的个数一定为奇数个,
    每次选择三个结点合并(减少两个结点),如此往复制止最后只剩一个结点,并返回这个结点。

    算法的输入$C$是点的集合,其中每个点包括三个儿子结点$childx$、$childy$、$childz$以及结点的频率
    $freq$。算法的返回值是最优解的三进制哈夫曼树。

    \newcommand{\Doo}{\>\textbf{}\hspace*{-0.7em}\'\addtocounter{indent}{1}}
    \begin{codebox}
        \Procname{$\proc{Huffman-Ternary}(C)$}
        \li $n = |C|$
        \li $Q = C$
        \li \If $n \% 2 == 0$ \Doo
        \li     allocate a new node $m$
        \li     $ m.childx = x = \proc{Extract-Min}(Q)$
        \li     $ m.childy = y = \proc{Extract-Min}(Q)$
        \li     $ m.childz = null$       
        \li     $ m.freq = x.freq + y.freq$
        \li     $ \proc{Insert}(Q,m)$ 
        \li     $ n = n - 1$\End
        \li \While $n > 1$\Doo
        \li     \If $n \% 2 == 0$ \Doo
        \li         allocate a new node $m$
        \li         $ m.childx = x = \proc{Extract-Min}(Q)$
        \li         $ m.childy = y = \proc{Extract-Min}(Q)$
        \li         $ m.childz = z = \proc{Extract-Min}(Q)$
        \li         $ m.freq = x.freq + y.freq + z.freq$
        \li         $ \proc{Insert}(Q,m)$ 
        \li         $ n = n - 2$ \End \End
        \li \Return $\proc{Extract-Min}(Q)$
    \end{codebox}

    \textit{最优解的证明:}
    
    $\proc{Huffman-Ternary}$贪心选择性的证明:

    欲证:字母表$C$,对于$c \in C$都有频率$c.freq$。若$x$、$y$、$z$是$C$中出现频率最小的
    三个字符,那么存在一个最优的三进制前缀码,$x$、$y$、$z$的码字长度相同,且只有最后一位不同。
    
    假定字母表$C$集合的大小是奇数,如果原来的字母表集合大小为偶数,则人为额外增加一个字母
    $\epsilon$,其出现频率$\epsilon.freq=0$。如果增加$\epsilon$之前的树为$T$,之后
    的树为$T'$,显然满足使得原花费$B(T)$最小的一组深度$d_T(c), c \in C$,
    也能使得$B(T') = \sum c.freq \cdot d_T(c) + 0 \cdot d_T(\epsilon)$最小,反之亦然。
  
    显然,在字母表集合为奇数的情况下,按照算法构造的三叉树的非叶子结点均包含三个儿子结点。
    假设$T$是$C$的一个最优三进制哈夫曼树,其中$a$,$b$,$c$是$T$中深度最深的三个兄弟叶子结点,
    希望将$x$、$y$、$z$与$a$,$b$,$c$交换后,得到的$T'$仍然是$C$的一个最优三进制哈夫曼树。

    不是一般性地假设$x \le y \le z$,$a \le b \le c$,首先交换结点$x$和$a$,得到树$T_1$,
    $T$和$T_1$的代价差$B(T)-B(T_1)$有:
    \resizebox{\linewidth}{!}{ 
        \begin{minipage}{380pt}
            \begin{align*}
                B(T)-B(T_1) &= \sum_{c\in C}c.freq \cdot d_T(c) - \sum_{c\in C}c.freq \cdot d_{T_1}(c)\\
                            &= x.freq \cdot d_T(x) + a.freq \cdot d_T(a) - x.freq \cdot d_T(x) - a.freq \cdot d_T(a)\\
                            &= (a.freq - x.freq) \cdot (d_T(a) - d_T(x))\\
                            &\ge 0
            \end{align*}
        \end{minipage}
    }
    ~\\
    同理,在$T_1$的基础上,再交换$y$和$b$,得到树$T_2$,则$T_1$和$T_2$的代价差亦有
    $B(T_1)-B(T_2)\ge 0$;在$T_2$的基础上,再交换$z$和$c$,得到树$T'$,则$T_2$和
    $T'$的代价差亦有$B(T_2)-B(T')\ge 0$。

    于是$B(T)-B(T')\ge 0$,显然交换后的解不会比原来解更坏,由于原来的解$T$是最优的,所以$T'$
    亦是最优的。
    
    $\proc{Huffman-Ternary}$最优子结构的证明:

    欲证:若$T'$是字母表$C'$的一个最优的三进制哈夫曼树,若将$T'$中的一个叶子结点$m'$,变为
    包含三个叶子结点$x$、$y$、$z$的中间结点$m$,要求
    $m.freq = m'.freq = x.freq + y.freq + z.freq$,且$x$、$y$、$z$的出现频率在新树$T$中最低,
    得到新字母表$C=C'-\{m'\}\cup\{x,y,z\}$对应的三进制哈夫曼树$T$也是最优的。

    尝试使用$B(T')$表示$B(T)$:
    \begin{align*}
        B(T)    &= B(T') + 1 \cdot m.freq\\
                &= B(T') + x.freq + y.freq + z.freq
    \end{align*}
    等价地:
    $$B(T') = B(T) - x.freq - y.freq - z.freq$$
    显然,如果$T$不是最优的,由贪心选择性结论,一定存在$T''$,包含$x$、$y$、$z$作为兄弟
    叶子结点,如果将$x$、$y$、$z$的父结点更换回$m'$得到树$T'''$,此时有:
    \begin{align*}
        B(T')   &= B(T) - x.freq - y.freq - z.freq \\
                &< B(T'') - x.freq - y.freq - z.freq \\
                &= B(T''')
    \end{align*}
    如此说明$T'$并不是字母表$C'$的一个最优的三进制哈夫曼树,产生矛盾,即$T$是字母表$C$
    最优的三进制哈夫曼树。

    综合贪心选择性和最优子结构性质,$\proc{Huffman-Ternary}$得证。
\end{solution}