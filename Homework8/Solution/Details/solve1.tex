\begin{solution}
    \textbf{a.证明:}对于图$G_f$,采用反证法,假设存在一条这样的环路径$u_1,u_2,\ldots,u_k,u_1$,显然这条环上必然存在一个相邻的结点
    $u_{s},u_{t}$,使得$u_{s}=\nu_s \wedge u_{t}=\nu_t\wedge s>t$,然而$G_f$中不存在任何$s>t$的$(\nu_{s},\nu_{t})$边,如此产生矛盾,
    即图$G_f$无环;由$E_f$可知,对于$\nu_1$,显然其不存在入边,即一定可以作为拓扑排序的首个元素。另一方面,对于每个元素$\nu_{p}$,其入边
    的初始结点$\nu_{q}$一定满足$q<p$,于是在拓扑序排完所有比$p$小的结点之后,$\nu_{p}$一定可以作为拓扑排序的下一个元素,于是归纳可得
    $\langle\nu_1,\nu_2,\ldots,\nu_{|V|}\rangle$ 是$G_f$的一个可行的拓扑排序。

    与之类似地,对于图$G_b$,采用反证法,假设存在一条这样的环路径$u_1,u_2,\ldots,u_k,u_1$,显然这条环上必然存在一个相邻的结点
    $u_{s},u_{t}$,使得$u_{s}=\nu_s \wedge u_{t}=\nu_t\wedge s<t$,然而$G_f$中不存在任何$s<t$的$(\nu_{s},\nu_{t})$边,如此产生矛盾,
    即图$G_f$无环;由$E_b$可知,对于$\nu_{|V|}$,显然其不存在入边,即一定可以作为拓扑排序的首个元素。另一方面,对于每个元素$\nu_{p}$,其入边
    的初始结点$\nu_{q}$一定满足$q>p$,于是在拓扑序排完所有比$p$大的结点之后,$\nu_{p}$一定可以作为拓扑排序的下一个元素,于是归纳可得
    $\langle\nu_{|V|},\nu_{|V|-1},\ldots,\nu_{1}\rangle$是$G_b$的一个可行的拓扑排序。

    \textbf{b.证明:}对于任意一条路径$\nu_{r_1},\nu_{r_2},\ldots,\nu_{r_k}$,如果$r_i$关于$i$单调增加,
    $G_f$根据无环且拓扑有序性质,必然这些点能够在同一个Relax轮次上确定最优值;利用$G_f$同理可以得到如果$r_i$关于$i$单调减小亦有
    如此的性质。进而,按照yen算法的流程,路径$\nu_{r_1},\nu_{r_2},\ldots,\nu_{r_s},\nu_{r_{s+1}},\ldots,\nu_{r_t}$可以在
    同一个Relax轮次上确定,当且仅当存在$r_1,r_2,\ldots,r_s$单调增加,$r_s,r_{s+1},\ldots,r_t$单调减小,这样的路径长度至少为2。\\
    因此这样对于任意一条最短路径$\nu_{w_1},\nu_{w_2},\ldots,\nu_{w_k}$,最终得到最优解所需要的Relax轮次等于$\lceil|V|/2\rceil$ 。

    \textbf{c.}由于仍然需要$O(\lceil|V|/2\rceil)$轮的Relax操作,且每轮操作时间花费均为$O(V)$,所以整体算法复杂度仍然为$O(VE)$。
\end{solution} 