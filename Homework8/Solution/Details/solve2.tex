\begin{solution}
    \textbf{a.}假设加入的新的边为$e(m,n)$。考虑结点$x,y$之间的连通性,若
    $(x,y)\notin E^* \wedge (x,y)\in (E\cup \{e\})^*$,则有$(x,m)\in E^* \wedge (n,y) \in E^*$。
    这样枚举每个结点$x,y$,若原先不连通,则检查$(x,m),(n,y)$的连通性,若均连通,则$(x,y)$属于新的传递闭包。
    整体算法的复杂度在于枚举的$O(V^2)$。

    \textbf{b.}假设图$G$的点构成的集合$V=\{v_1,v_2,\ldots,v_n\}$,边构成的集合为
    $E=\{(v_{x},v_{x+1})\mid 1 \le x \le n-1\}$,则$E^*=\{(v_{x},v_{y})\mid 1\le x \le y\le n\}$。
    若加入新边为$e(v_n,v_1)$,则$(E\cup\{e\})^*=\{(v_{x},v_{y})\mid 1\le x,y\le n\}$。
    易得,$|E^*|=\frac{|V|^2+|V|}{2}$,$|(E\cup\{e\})^*|=|V|^2$,于是加入$e$之后需要更新的边的数目为
    $\frac{|V|^2-|V|}{2}=\Omega(|V|^2)$。

    \textbf{c.}对于每个结点维护一棵树,树上点结点是$E$中所有可以到达这个结点的点,维护另一棵树,树上点结点是$E^*$
    中所有这个结点可以到达的点的传递闭包。每次加一条边$(u,v)$,即将$v$插入到$u$及其所有祖先结点(递归访问前驱结点,
    如果已经存在传递闭包则可停止)。对于每条边$e$,至多重复访问$|V|$次,故求和的复杂度为$O(V^3)$。

\end{solution}