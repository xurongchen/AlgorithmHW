\textit{\textbf{Yen's improvement to Bellman-Ford}}

Suppose that we order the edge relaxations in each pass of the Bellman-Ford algorithm as follows. 
Before the first pass, we assign an arbitrary linear order $\nu_1,\nu_2,\ldots,\nu_{|V|}$ to the 
vertices of the input graph $G=(V,E)$. Then, we partition the edge set $E$ into $E_f \cup E_b$, 
where $E_f = \{(\nu_i,\nu_j)\in E: i < j\}$ and $E_b = \{(\nu_i,\nu_j)\in E: i > j\}$. (Assume 
that $G$ contains no self-loops, so that every edge is in either $E_f$ or $E_b$.) Define $G_f=(V,E_f)$ 
and $G_b=(V,E_b)$.

\textbf{a.} Prove that $G_f$ is acyclic with topological sort $\langle\nu_1,\nu_2,\ldots,\nu_{|V|}\rangle$ 
and that $G_b$ is acyclic with topological sort $\langle\nu_{|V|},\nu_{|V|-1},\ldots,\nu_{1}\rangle$.

Suppose that we implement each pass of the Bellman-Ford algorithm in the following way. We visit each 
vertex in the order $\nu_1,\nu_2,\ldots,\nu_{|V|}$, relaxing edges of $E_f$ that leave the vertex. 
We then visit each vertex in the order $\nu_{|V|},\nu_{|V|-1},\ldots,\nu_{1}$, relaxing edges of $E_b$ 
that leave the vertex.

\textbf{b.} Prove that with this scheme, if $G$ contains no negative-weight cycles that are reachable from the
 source vertex $s$, then after only $\lceil|V|/2\rceil$ passes over the edges, $\nu.d = \delta(s,\nu$ 
 for all vertices $\nu \in V$.

\textbf{c.} Does this scheme improve the asymptotic running time of the Bellman-Ford algorithm?
