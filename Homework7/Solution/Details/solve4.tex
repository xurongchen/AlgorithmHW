\begin{solution}
    \textbf{算法设计:}整体的算法如$\proc{Get-Negative-Cycle}$所示。首先对图$G$计算其全连通分量,
    时间复杂度$O(V+E)$。对于每一个全连通分量的子图$C_i$,执行$\proc{Improved-Bellman-Ford}$
    判断是否该全连通分量子图上是否有负权重的环(由于全连通分量保证内部点的连通性,可任意选取点作为源点),
    如果存在负权重的环,利用各点的最短路的前驱结点信息进行深度优先遍历,找到负权重的环中的某一个点,最后
    在利用这个已知环上的点得到整个负权重环。执行Bellman-Ford算法的总时间之和为
    $O(\Sigma V_iE_i) \le O(VE)$,深度优先遍历求负权重环的时间复杂度为$O(V+E)$。\\
    综上所述,算法的整体复杂度为$O(VE)$。
    \newcommand{\Doo}{\>\textbf{}\hspace*{-0.7em}\'\addtocounter{indent}{1}}
    \resizebox{\linewidth}{!}{ 
        \begin{minipage}{330pt}
            \begin{codebox}
                \Procname{$\proc{Get-Negative-Cycle}(G,w)$}
                \li $G^{SCC}=\proc{Strongly-Connercted-Components}(G)$
                \li \For $C_i$ in $G^{SCC}.V$
                    \Doo
                \li     Let $r$ be a empty List
                \li     Let $s_i$ be an random node in $C_i$
                \li     \If not $\proc{Improved-Bellman-Ford}(C_i,w,s_i,r)$
                        \Doo
                \li         \Return $r$
                        \End
                    \End
                \li \textbf{throw} \textbf{exception:} \textit{No negative cycle}
            \end{codebox}
        \end{minipage}
    } 

    $\proc{Improved-Bellman-Ford}$与一般性的Bellman-Ford算法的区别在于以下两个方面:\\
    第一,$\proc{Relax}$的次数从
    $|G.V|-1$次变为$|G.V|$次,这样保证如果存在负权重的环,图中所有点$p$以及其前驱$p.\pi$形成的边$(p,p.\pi)$
    一定能形成上述的负权重环;\\
    第二,除了返回是否存在负权重的环的结果(存在返回False),通过$\proc{Get-Negative-Cycle}$的调用,
    参数$r$能够同时记录负权重环的所有结点。
    \resizebox{\linewidth}{!}{
        \begin{minipage}{330pt}
            \begin{codebox}
                \Procname{$\proc{Improved-Bellman-Ford}(G,w,s,r)$}
                \li $\proc{Initialize-Single-Source}(G,s)$
                \li \For $i=1$ \To $|G.V|$
                    \Doo
                \li     \For edge$(u,v)$ in $G.E$
                        \Doo
                \li         $\proc{Relax}(u,v,w)$
                        \End
                    \End
                \li \For edge$(u,v)$ in $G.E$
                    \Doo
                \li     \If $v.d > u.d + w(u,v)$
                        \Doo
                \li         $r = \proc{Get-Negative-Cycle}(G)$
                \li         \Return False
                        \End
                    \End
                \li \Return True
            \end{codebox}
        \end{minipage}
    }

    $\proc{Get-Negative-Cycle}$首先初始化访问状态数组$S$(访问状态包括$\textsf{ToVisit}$、$\textsf{Visiting}$
    以及$\textsf{Visited}$)。由于结点和其对应的前驱能够形成一个环(即所求的负权重环),对于所有结点进行深度优先遍历$\proc{DFS}$,
    遍历过程之中同时更新各个结点的访问状态。$\proc{DFS}$过程之中,若发现某个结点的状态为$\textsf{Visiting}$,则说明该结点一定
    在所求的负权重环之上,则返回该结点,并不断利用结点的前驱信息,构建整个负权重环。

    \resizebox{\linewidth}{!}{
        \begin{minipage}{330pt}
            \begin{codebox}
                \Procname{$\proc{Get-Negative-Cycle}(G)$}
                \li Initialize $|G.V|$ size of array $S$ with $\textsf{ToVisit}$
                \li \For $n$ in $G.V$
                    \Doo
                \li     $y = \proc{DFS}(n,G,S)$
                \li     \If $y \neq$ null
                        \Doo
                \li         Let $r$ be a empty List
                \li         $r.$add$(y)$
                \li         $x = y.\pi$
                \li         \While $x\neq y$
                            \Doo
                \li             $r.$add$(x)$
                \li             $x = x.\pi$
                            \End
                \li         \Return $r$
                        \End
                    \End
                \li \textbf{throw} \textbf{exception:} \textit{Get cycle failed}
            \end{codebox}
        \end{minipage}
    }
    \resizebox{\linewidth}{!}{ 
        \begin{minipage}{330pt}
            \begin{codebox}
                \Procname{$\proc{DFS}(n,G,S)$}
                \li \If $S[n] == \textsf{Visited}$ or $n.\pi ==$ null
                    \Doo
                \li     \Return null
                    \End
                \li \If $S[n] == \textsf{Visiting}$
                    \Doo
                \li     \Return $n$
                    \End
                \li $S[n] = \textsf{Visiting}$
                \li $u = \proc{DFS}(n.\pi,G,S)$
                \li $S[n] = \textsf{Visited}$
                \li \Return u
            \end{codebox}
        \end{minipage}
    }

    % \textbf{复杂度分析:}

    \textbf{证明:}
    证明分为以下3个步骤:\\
    \textit{步骤1:}如果图$G$存在负权重的环,则对于$G$的某一个强连通分量$C_t$,并从$C_t$中任意选取结点$S_t\in C_t.V$执行
    Bellman-Ford算法,则一定可以从中发现负权重的环的存在;\\

    \qquad 对于$G$内一个负权重环,显然环上的所有点是连通的,即对$G$进行强连通分量的划分,这个负权重环上的所有点在同一个强连通
    分量之上。如果Bellman-Ford算法返回False,则说明从源点出发存在到负权环的路径,而对于强连通分量,由于内部点是相互连通的,故
    如果强连通分量内部存在负权环,则从任意点出发,Bellman-Ford算法均可返回False。

    \textit{步骤2:}对于存在负权环的强连通子图$C_t$,对所有边执行$|G.V|$次$\proc{Relax}$操作得到原$C_t$的一个子图$M$
    仍包含该负权环且不包含正权环,其中$M.V=C_t.V$,$M.E=\{edge(x,y) \mid y = x.\pi\wedge x,y \in C_t.V \}$;\\
    
    \qquad 根据结点的前驱$\pi$的含义可知,执行充分次数$\proc{Relax}$操作之后,$\pi$可以稳定。而因为环的长度一定不超过图中点的个数,因此至多执行
    $|C_t.V|$次之后所有结点的前驱$\pi$可以稳定,并且对于负权环上的任意一个点,其前驱一定是其在环上的前驱。同时,$\proc{Relax}$无论执行多少次数
    显然都不会引入正权重的环,于是若强连通子图$C_t$存在负权环,对应得到的$C_t$的子图$M$唯一地包含一个环即原$C_t$所包含的负权环。

    \textit{步骤3:}对于子图$M$,如果$M$存在唯一的环即负权重的环,那么执行$\proc{Get-Negative-Cycle}$,可以得到负权重
    环上的所有元素。\\

    \qquad 对于$M$执行深度优先遍历,由于$M$存在环,显然若在某一结点$x$的前驱闭包之中发现$x$,说明$x$一定在环上(对应深度优先
    遍历之中访问结点时发现结点的状态是$\textsf{Visiting}$)。对于环上任意的一个元素$x$,不断访问其前驱直至又一次遇到$x$
    停止,显然可以正确地得到环上的所有元素。

    综上所述,由步骤1、2、3,可以证明出算法的正确性。
\end{solution}