\begin{solution}
    \textbf{算法设计:}算法如\proc{Is-Semiconnected}所示。首先对原图求解强连通分量,
    得到$G$的分量图$G^{SCC}$,显然$G^{SCC}$是一个有向无环图。于是对$G^{SCC}$进行拓扑
    排序,得到可行序列$T$。遍历$T$,若相邻结点对在$G^{SCC}$中无边,则原图非半连通图,若所有
    相邻结点均有边则原图为半连通图。
    \newcommand{\Doo}{\>\textbf{}\hspace*{-0.7em}\'\addtocounter{indent}{1}}
    \resizebox{\linewidth}{!}{ 
    \begin{minipage}{330pt}
        \begin{codebox}
            \Procname{$\proc{Is-Semiconnected}$($G$)}
            \li $G^{SCC}(V^{SCC},E^{SCC})=\proc{Strongly-Connercted-Components}(G)$
            \li $T = \proc{Topological-Sort}(G^{SCC})$
            \li $node = T.first$
            \li \While $node.next \neq$ null
                \Doo 
            \li \If $(node,node.next)$ not in $E^{SCC}$ 
                \Doo
            \li     \Return False
                \End
            \li     $node = node.next$
                \End
            \li \Return True
        \end{codebox}
    \end{minipage}
    }

    \textbf{证明:}由下面的引理1、引理2易得算法的正确性。

    \textit{引理1:}$G$和分量图$G^{SCC}$的半连通性一致。

    证明:(1)$G$半连通到$G^{SCC}$半连通的证明:要证在$G^{SCC}$中任意的点$C_u,C_v$是半连通的,即要证$G$中存在点
    $u\in C_u$,$v\in C_v$,$u \leadsto v$或$v \leadsto u$成立。而$G$的半连通性质则说明了对于$G$中任意两点都有
    $u \leadsto v$或$v \leadsto u$成立,因此在$G^{SCC}$中$C_u\leadsto C_v$成立。\\
    (2)$G^{SCC}$半连通到$G$半连通的证明:对于任意的$u,v\in V$,不妨设其所在的连通分量分别为$C_u,C_v$。\\
    \qquad $\circ$ 若$C_u = C_v$:由连通分量定义可得$u \leadsto v$。\\
    \qquad $\circ$ 若$C_u \neq C_v$:不是一般性地假设在$G^{SCC}$中$C_u \leadsto C_v$。于是在$G$中必然存在
    $\mathfrak{u}\in C_u$,$\mathfrak{v}\in C_v$,使得$\mathfrak{u} \leadsto \mathfrak{v}$。又由于
    $u$和$\mathfrak{u}$均在连通分量$C_u$中,故$u \leadsto \mathfrak{u}$;同理,$\mathfrak{v} \leadsto v$。
    于是$u \leadsto v$。
    \medskip

    \textit{引理2:}若分量图$G^{SCC}$的拓扑序$T[x_1,\cdots,x_n]$的任意相邻结点$x_k$,$x_{k+1}$在
    $E^{SCC}$中有边$(x_k,x_{k+1})$,则$G^{SCC}$是半连通图。反之亦然。

    证明:(1)拓扑序到半连通的证明:对于任意两个结点$x_u,x_v$,不是一般性地设定$u<v$,非常显然地存在这样的路径
    $x_u,x_{u+1},\cdots,x_v$使得$x_u\leadsto x_v$,命题正向得证。\\
    (2)半连通到拓扑序的证明:采用反证法,假设存在半连通图$G^{SCC}$,其分量图拓扑序列为$T[x_1,\cdots,x_n]$中存
    在使得$(x_k,x_{k+1}) \notin E^{SCC}$成立的$x_k$,$x_{k+1}$。由于半连通性的要求,$x_k \leadsto x_{k+1}$
    或$x_{k+1} \leadsto x_{k}$中至少一个成立。\\
    $\bullet$ 若$x_k \leadsto x_{k+1}$成立,因为$(x_k,x_{k+1}) \notin E^{SCC}$,于是至少存在一个中间结点
    $x_t$,$x_k \leadsto x_t \leadsto x_{k+1}$。\\
    \qquad $\circ$ 若$t<k$:显然子路径$x_k \leadsto x_t$使得$x_k$必然属于$x_t$的前驱闭包,即拓扑序之中$x_k$必然在
    $x_t$之前出现,这与$t<k$矛盾,故$t<k$的情况不成立。\\
    \qquad $\circ$ 若$t>k+1$:显然子路径$x_{t} \leadsto x_{k+1}$使得$x_t$必然属于$x_{k+1}$的前驱闭包,即拓扑
    序之中$x_t$必然在$x_{k+1}$之前出现,这与$t>k+1$矛盾,故$t>k+1$的情况不成立。\\
    \qquad $\circ$ 上述两点可知$x_k \leadsto x_{k+1}$不成立。\\
    $\bullet$ 若$x_{k+1} \leadsto x_k$成立,则$x_{k+1}$必然属于$x_{k}$的前驱闭包,即拓扑序之中$x_{k+1}$必然在
    $x_{k}$之前出现,这与$k<k+1$矛盾,故$x_{k+1} \leadsto x_k$不成立。

    综上反证假设不成立,命题反向得证。

    \textbf{复杂度分析:}求解强连通分量的时间复杂度为$O(|V|+|E|)$;求解拓扑的复杂度为
    $O(|V^{SCC}|+|E^{SCC}|)$,最坏不超过$O(|V|+|E|)$;判断相邻结点对在$G^{SCC}$中
    是否有边复杂度亦为$O(|V^{SCC}|+|E^{SCC}|)$(对边建哈希表$O(|E^{SCC}|)$,遍历结点次数
    为$O(|V^{SCC}|)$,每次$O(1)$。综上\proc{Is-Semiconnected}算法的复杂度为$O(|V|+|E|)$。
\end{solution} 