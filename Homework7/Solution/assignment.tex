% Homework template for Algorithm Analysis and Design
% UPDATE: September 20, 2019 by Xu Rongchen
\documentclass[a4paper]{article}
\usepackage{ctex}
\ctexset{
proofname = \heiti{证明} %% set proof name
}
\usepackage{amsmath, amssymb, amsthm}
% amsmath: equation*, amssymb: mathbb, amsthm: proof
\usepackage{moreenum}
\usepackage{mathtools}
\usepackage{url}
\usepackage{bm}
\usepackage{enumitem}
\usepackage{graphicx}
\usepackage{subcaption}
\usepackage{booktabs} % toprule
\usepackage[mathcal]{eucal}

\usepackage[thehwcnt = 6]{iidef} % set homework count
\usepackage{longtable}

% \usepackage[noend]{algpseudocode}
\usepackage{clrscode3e}

\thecoursename{算法分析与设计}
\theterm{2019年秋季学期}
\hwname{作业}
\slname{\heiti{解}}
\begin{document}
\courseheader
\theusername{徐荣琛}
\thestuno{2019214518}
\theinstitute{软件学院}

\info

\begin{enumerate}
  \setlength{\itemsep}{3\parskip}
  %% Homework Start here:
  %% \item to enumerate the problem ID: Format as 'HomeworkID.ProblemID'
  %% \begin{solution} XXXX \end{solution} is to make a solution
  %% \begin{proof} XXXX \end{proof} is to make a proof
  %% Suggest to use \input{path} command
  \item Prove: $2n+\Theta(n^2) = \Theta(n^2)$.
  \begin{solution}
    首先$O(m)$地计算$P$的前缀函数$\pi$。
    然后初始化$\delta$为0,再外层从$0$到$m$循环$q$,内层循环$\Sigma$中每个字母$a$,若
    \begin{enumerate}
        \item $q < m \wedge P[q+1] = a$,则$\delta(q,a) = q+1$;
        \item $q = m \vee P[q+1] \neq a$,则$\delta(q,a) = \delta(\pi(q),a)$;
    \end{enumerate}
    \begin{proof}

            % (a) 根据自动机状态的定义,该式显然成立;
            % (b) 
            
            根据自动机状态的定义,自动机状态编号即为相同的$T$的后缀在$P$前缀上匹配的最大长度,
            因此假设集合$F(q,a)=\{k\mid P_k \sqsupset P_q \wedge P[k+1] = a\}$ 
            我们希望证明:
            $$\delta(q,a) = \max_k F(q,a) \text{ if } F \neq \emptyset \text{ else } 0$$
            
            对$q$进行归纳:
            
            (1)初始,$\delta(0,x)=0 \text{ if } x \neq P[1] \text{ else } 1$,由
            (a)和初始化条件,显然成立。
            
            (2) 假设对$q<q_0$成立,考虑$F(q,a)$的等价表示:
            $F(q,a)=\{k\mid P_k \sqsupset P_q \wedge P[k+1] = a\} = \{k\mid k \in \pi^*(q) \wedge P[k+1] = a\}$

            若$q < m \wedge P[q+1] = a$,显然$\max F(q,a) = q$,于是$\delta(q,a) = q+1$;
            若$q = m \vee P[q+1] \neq a$,显然$q \notin F(q,a)$,于是$F(q,a) = \max\{k\mid k \in \pi^*(q) \wedge P[k+1] = a\}=\max\{k\mid k \in \pi^*(\pi(q)) \wedge P[k+1] = a\}=\delta(\pi(q),a)$

           综上(a)、(b)得证。
            % 根据$q$的长度进行归纳:
            % % $\delta(\pi(q),a) = \pi(q)+1$
            
            % 初始$q=0$,显然成立;\\
            % % 假设$q<q_0$成立:
            % 由于$P_{\pi(q)}\sqsupset P_q$,于是$P_{\pi(q)}a\sqsupset P_qa$,又
            % $\delta(q,a) = \sigma(P_qa) = \max\{k\mid P_k \sqsupset P_qa\}$。
            % 因此,$\delta(\pi(q),a)\in \{k\mid P_k \sqsupset P_qa\}$。

            % 下面证明是最大的。

            % 由于以$a$结尾,不妨设存在另一个最大的$\delta(l,a)>\delta(\pi(q),a)$,且$P_la \sqsupset P_qa$。
            % 显然,$l$满足$l\in\{l\mid l<q \wedge P_l\sqsupset P_q\}$,
            % 由于$l>\delta(l,a)-1>\delta(\pi(q),a)$

            % 于是$\pi(q) = \max\{t\mid t<q \wedge P_t\sqsupset P_q\} = \max\{t\mid t<q \wedge P_ta\sqsupset P_qa\} = $
            
            % 由于$q = m \vee P[q+1] \neq a$,故$\sigma(P_qa) < P_q + 1 = q + 1$。
            % 而$\pi(q) = \max\{t\mid t<q \wedge P_t\sqsupset P_q\} = \max\{t\mid t<q \wedge P_ta\sqsupset P_qa\} = $,
            % 假设最值为$t_m$,
            % 则$P_{t_m}\sqsupset P_q$
    \end{proof}
\end{solution}
  \item As stated, in dynamic programming we first solve the subproblems and then 
choose which of them  to use in an optimal solution to the problem. 
Professor Capulet claims that we do not always need to solve all the subproblems 
in order to find an optimal solution. She suggests that we can find an optimal 
solution to the matrix-chain multiplication problem by always choosing the 
matrix $A_k$ at which to split the subproduct $A_iA_{i+1}\cdots A_j$
(by selecting $k$ to minimize the quantity $p_{i-1}p_kp_j$) before solving the 
subproblems. Find an instance of the matrix-chain multiplication problem 
for which this greedy approach yields a suboptimal solution.
  \begin{solution}
    若要求点在桶中实现均匀分布,则需要将单位圆按面积等分为$n$个同心圆环(最内部为圆)。\\
    $$\pi r_0^2 \le \frac{\pi}{n}$$
    $$\frac{i\pi}{n} < \pi r_i^2 \le \frac{(i+1)\pi}{n}, i\in [1,n-1]$$
    即$ nr_0^2 \le 1$,$ i < nr_i^2 \le i+1$\\
    按$n(x^2+y^2)$为关键字进行桶排序,$n$个桶为$[0,1],(1,2],(2,3],\cdots,$ $(n-1,n]$.
    这样可得到$\Theta(n)$复杂度的排序算法.
\end{solution}  
  \item \textbf{\textbf{Minimun path cover}}

A path cover of a directed graph $G=(V,E)$ is a set $P$ of vertex-disjoint 
paths such that every vertex in $V$ is included in exactly one path in $P$. 
Paths may start and end anywhere, and they may be of any length, including 0. 
A \textbf{Minimun path cover} of $G$ is a path cover containing the fewest 
possible paths.

\textbf{a.} Give an efficient algorithm to find a minimum path cover of a 
directed acyclic graph $G=(V,E)$. (Hint: Assuming that $V=\{1,2,\ldots,n\}$, 
construct the graph $G'=(V',E')$,where\\
$V' =\{x_0,x_1,\ldots,x_n\}\cup \{y_0,y_1,\ldots,y_n\}$,\\
$E' =\{(x_0,x_i):i\in V\}\cup \{(y_i,y_0):i\in V\}\cup (x_i,y_j):(i,j)\in E$,\\
and run a maximum-flow algorithm.)

\textbf{b.} Does your algorithm work for directed graphs that contain cycles? Explain.
  \begin{solution}    
    % \newcommand{\Doo}{\>\textbf{}\hspace*{-0.7em}\'\addtocounter{indent}{1}}
    % \resizebox{\linewidth}{!}{ 
    %     \begin{minipage}{380pt}
    %         \begin{codebox}
    %             \Procname{$\proc{Minimun-Path-Cover}(G)$}
    %             \li $G'=(V',E')$ \\\Doo
    %                 \textit{in which:} $V' =\{x_0,x_1,\ldots,x_n\}\cup \{y_0,y_1,\ldots,y_n\}$,\\\Doo
    %                 $E' =\{(x_0,x_i):i\in V\}\cup \{(y_i,y_0):i\in V\}\cup (x_i,y_j):(i,j)\in E$
    %                 \End \End
    %             \li $\forall (u,v) \in E'$ set the capacity $G'.c(u,v) = 1$
    %             \li $\proc{Max-Flow}(G,x_0,y_0)$
    %             \li \

    %             \li \For each vertex $u\in G.V - \{s,t\}$\Doo
    %             \li     $u.current = u.N.head$
    %                 \End
    %             \li $L = $ ordered linked list of vertex in $G$ by height decreasing
    %             \li $u = L.Head$
    %             \li \While $u\neq$ NIL \Doo
    %             \li     $old\_height = u.Height$
    %             \li     $\proc{Discharge}(u)$
    %             \li     \If $old\_height < u.height$ \Doo
    %             \li         Forward swap $u$ with the former node until the order is correct
    %             \li     \Else 
    %             \li         $u = u.Next$
    %         \end{codebox}
    %     \end{minipage}
    % }
    \textbf{a.} 对于图G',对于每条边的容量设置为1,执行从$x_0$到$y_0$的最大流算法。对于得到的最大流$f$,如果$f(x_i,y_j)=1$,即在原图
    $G$之中,边$(i,j)$存在于最小路径覆盖集合中的某一个路径之上。

    正确性证明:在不考虑环的情况下,最小路径覆盖必然对应选出最多的边,且这些边没有共同的初始结点或没有结束的初始结点。
    首先$f(x_0,x_i)=1$保证了没有共同的初始结点,对应的$f(y_j,y_0)=1$保证了没有共同的结束结点。于是要使得选出最多的边,即求
    图$G'$中最大的流。

    \textbf{b.} 不能适用于带环的图。存在$G$满足$V=\{1,2,3\}$,$E=\{(1,2),(2,$ $1),(2,3)\}$。显然$G'$一个可行的最大流
    包括$(x_0,x_1),(x_0,x_2),(x_1,y_2),$ $(x_2,y_1),(y_1,y_0),(y_2,y_0)$。显然对应选出的边为$(1,2),(2,1)$,并不能够同时
    属于一个可行的最小路径覆盖,因为存在结点成环的情况。

\end{solution}  
  \item Suppose that a weighted, directed graph $G = (V,E)$ has a negative-
weight cycle. Give an efficient algorithm to list the vertices of one 
such cycle. Prove that your algorithm is correct.

  \begin{solution}
    \textbf{算法设计:}整体的算法如$\proc{Get-Negative-Cycle}$所示。首先对图$G$计算其全连通分量,
    时间复杂度$O(V+E)$。对于每一个全连通分量的子图$C_i$,执行$\proc{Improved-Bellman-Ford}$
    判断是否该全连通分量子图上是否有负权重的环(由于全连通分量保证内部点的连通性,可任意选取点作为源点),
    如果存在负权重的环,利用各点的最短路的前驱结点信息进行深度优先遍历,找到负权重的环中的某一个点,最后
    在利用这个已知环上的点得到整个负权重环。执行Bellman-Ford算法的总时间之和为
    $O(\Sigma V_iE_i) \le O(VE)$,深度优先遍历求负权重环的时间复杂度为$O(V+E)$。\\
    综上所述,算法的整体复杂度为$O(VE)$。
    \newcommand{\Doo}{\>\textbf{}\hspace*{-0.7em}\'\addtocounter{indent}{1}}
    \resizebox{\linewidth}{!}{ 
        \begin{minipage}{330pt}
            \begin{codebox}
                \Procname{$\proc{Get-Negative-Cycle}(G,w)$}
                \li $G^{SCC}=\proc{Strongly-Connercted-Components}(G)$
                \li \For $C_i$ in $G^{SCC}.V$
                    \Doo
                \li     Let $r$ be a empty List
                \li     Let $s_i$ be an random node in $C_i$
                \li     \If not $\proc{Improved-Bellman-Ford}(C_i,w,s_i,r)$
                        \Doo
                \li         \Return $r$
                        \End
                    \End
                \li \textbf{throw} \textbf{exception:} \textit{No negative cycle}
            \end{codebox}
        \end{minipage}
    } 

    $\proc{Improved-Bellman-Ford}$与一般性的Bellman-Ford算法的区别在于以下两个方面:\\
    第一,$\proc{Relax}$的次数从
    $|G.V|-1$次变为$|G.V|$次,这样保证如果存在负权重的环,图中所有点$p$以及其前驱$p.\pi$形成的边$(p,p.\pi)$
    一定能形成上述的负权重环;\\
    第二,除了返回是否存在负权重的环的结果(存在返回False),通过$\proc{Get-Negative-Cycle}$的调用,
    参数$r$能够同时记录负权重环的所有结点。
    \resizebox{\linewidth}{!}{
        \begin{minipage}{330pt}
            \begin{codebox}
                \Procname{$\proc{Improved-Bellman-Ford}(G,w,s,r)$}
                \li $\proc{Initialize-Single-Source}(G,s)$
                \li \For $i=1$ \To $|G.V|$
                    \Doo
                \li     \For edge$(u,v)$ in $G.E$
                        \Doo
                \li         $\proc{Relax}(u,v,w)$
                        \End
                    \End
                \li \For edge$(u,v)$ in $G.E$
                    \Doo
                \li     \If $v.d > u.d + w(u,v)$
                        \Doo
                \li         $r = \proc{Get-Negative-Cycle}(G)$
                \li         \Return False
                        \End
                    \End
                \li \Return True
            \end{codebox}
        \end{minipage}
    }

    $\proc{Get-Negative-Cycle}$首先初始化访问状态数组$S$(访问状态包括$\textsf{ToVisit}$、$\textsf{Visiting}$
    以及$\textsf{Visited}$)。由于结点和其对应的前驱能够形成一个环(即所求的负权重环),对于所有结点进行深度优先遍历$\proc{DFS}$,
    遍历过程之中同时更新各个结点的访问状态。$\proc{DFS}$过程之中,若发现某个结点的状态为$\textsf{Visiting}$,则说明该结点一定
    在所求的负权重环之上,则返回该结点,并不断利用结点的前驱信息,构建整个负权重环。

    \resizebox{\linewidth}{!}{
        \begin{minipage}{330pt}
            \begin{codebox}
                \Procname{$\proc{Get-Negative-Cycle}(G)$}
                \li Initialize $|G.V|$ size of array $S$ with $\textsf{ToVisit}$
                \li \For $n$ in $G.V$
                    \Doo
                \li     $y = \proc{DFS}(n,G,S)$
                \li     \If $y \neq$ null
                        \Doo
                \li         Let $r$ be a empty List
                \li         $r.$add$(y)$
                \li         $x = y.\pi$
                \li         \While $x\neq y$
                            \Doo
                \li             $r.$add$(x)$
                \li             $x = x.\pi$
                            \End
                \li         \Return $r$
                        \End
                    \End
                \li \textbf{throw} \textbf{exception:} \textit{Get cycle failed}
            \end{codebox}
        \end{minipage}
    }
    \resizebox{\linewidth}{!}{ 
        \begin{minipage}{330pt}
            \begin{codebox}
                \Procname{$\proc{DFS}(n,G,S)$}
                \li \If $S[n] == \textsf{Visited}$ or $n.\pi ==$ null
                    \Doo
                \li     \Return null
                    \End
                \li \If $S[n] == \textsf{Visiting}$
                    \Doo
                \li     \Return $n$
                    \End
                \li $S[n] = \textsf{Visiting}$
                \li $u = \proc{DFS}(n.\pi,G,S)$
                \li $S[n] = \textsf{Visited}$
                \li \Return u
            \end{codebox}
        \end{minipage}
    }

    % \textbf{复杂度分析:}

    \textbf{证明:}
    证明分为以下3个步骤:\\
    \textit{步骤1:}如果图$G$存在负权重的环,则对于$G$的某一个强连通分量$C_t$,并从$C_t$中任意选取结点$S_t\in C_t.V$执行
    Bellman-Ford算法,则一定可以从中发现负权重的环的存在;\\

    \qquad 对于$G$内一个负权重环,显然环上的所有点是连通的,即对$G$进行强连通分量的划分,这个负权重环上的所有点在同一个强连通
    分量之上。如果Bellman-Ford算法返回False,则说明从源点出发存在到负权环的路径,而对于强连通分量,由于内部点是相互连通的,故
    如果强连通分量内部存在负权环,则从任意点出发,Bellman-Ford算法均可返回False。

    \textit{步骤2:}对于存在负权环的强连通子图$C_t$,对所有边执行$|G.V|$次$\proc{Relax}$操作得到原$C_t$的一个子图$M$
    仍包含该负权环且不包含正权环,其中$M.V=C_t.V$,$M.E=\{edge(x,y) \mid y = x.\pi\wedge x,y \in C_t.V \}$;\\
    
    \qquad 根据结点的前驱$\pi$的含义可知,执行充分次数$\proc{Relax}$操作之后,$\pi$可以稳定。而因为环的长度一定不超过图中点的个数,因此至多执行
    $|C_t.V|$次之后所有结点的前驱$\pi$可以稳定,并且对于负权环上的任意一个点,其前驱一定是其在环上的前驱。同时,$\proc{Relax}$无论执行多少次数
    显然都不会引入正权重的环,于是若强连通子图$C_t$存在负权环,对应得到的$C_t$的子图$M$唯一地包含一个环即原$C_t$所包含的负权环。

    \textit{步骤3:}对于子图$M$,如果$M$存在唯一的环即负权重的环,那么执行$\proc{Get-Negative-Cycle}$,可以得到负权重
    环上的所有元素。\\

    \qquad 对于$M$执行深度优先遍历,由于$M$存在环,显然若在某一结点$x$的前驱闭包之中发现$x$,说明$x$一定在环上(对应深度优先
    遍历之中访问结点时发现结点的状态是$\textsf{Visiting}$)。对于环上任意的一个元素$x$,不断访问其前驱直至又一次遇到$x$
    停止,显然可以正确地得到环上的所有元素。

    综上所述,由步骤1、2、3,可以证明出算法的正确性。
\end{solution}

\end{enumerate}


\end{document}